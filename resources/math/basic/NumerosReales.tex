\documentclass{article}
\usepackage[utf8]{inputenc}
\usepackage{amsmath, amssymb}

\title{Números Reales y sus Propiedades}
\author{GustavoJhon}
\date{}

\begin{document}

\maketitle

El conjunto de los \textbf{números reales} (denotado por $\mathbb{R}$) incluye tanto a los números racionales (positivos, negativos y el cero) como a los números irracionales. Dentro de este conjunto, también podemos clasificarlos en \textbf{trascendentes} y \textbf{algebraicos}.

\section{Sistema de Números Reales}

El sistema de números reales se compone principalmente de dos grandes conjuntos:

\begin{itemize}
    \item \textbf{Números racionales} 
    \item \textbf{Números irracionales}
\end{itemize}

\subsection{Números Irracionales ($\mathbb{I}$)}

Un número irracional es aquel que \textbf{no puede expresarse como una fracción} $\frac{m}{n}$, donde $m$ y $n$ son enteros y $n \neq 0$. Algunos ejemplos de números irracionales incluyen:

\begin{itemize}
    \item \textbf{Raíces} cuadradas no exactas, como $\sqrt{2}$
    \item Números trascendentes como el número $\pi$ y $e$
\end{itemize}

\subsection{Números Racionales ($\mathbb{Q}$)}

Un número racional es aquel que \textbf{puede expresarse como una fracción} $\frac{a}{b}$, donde $a$ y $b$ son enteros y $b \neq 0$. Dentro de los números racionales encontramos:

\begin{itemize}
    \item \textbf{Fracciones} comunes, como $\frac{3}{4}$
    \item \textbf{Decimales} finitos y \textbf{decimales periódicos} (ejemplo: $0.333\ldots = \frac{1}{3}$)
\end{itemize}

\subsection{Números Naturales ($\mathbb{N}$)}

Los números naturales incluyen los números \textbf{positivos} que usamos para contar:

\[
\mathbb{N} = \{1, 2, 3, 4, \ldots\}
\]

\subsection{Números Enteros ($\mathbb{Z}$)}

El conjunto de los números enteros contiene a los \textbf{números naturales}, sus \textbf{inversos aditivos} (números negativos), y el \textbf{cero}:

\[
\mathbb{Z} = \{\ldots, -3, -2, -1, 0, 1, 2, 3, \ldots\}
\]

\section{Operaciones con Números Reales}

Con los números reales se pueden realizar operaciones básicas, aunque con algunas \textbf{restricciones importantes}:

\begin{itemize}
    \item \textbf{Raíces de orden par de números negativos} no están definidas en los números reales.
    \item La \textbf{división entre cero} no está definida.
    \item No se puede calcular el \textbf{logaritmo de un número real negativo}.
\end{itemize}

\section{Propiedades de los Números Reales}

A continuación, se presentan las propiedades de los números reales en las operaciones de \textbf{suma} y \textbf{multiplicación}.

\begin{center}
\begin{tabular}{|c|c|c|}
\hline
\textbf{Propiedad} & \textbf{Suma} & \textbf{Multiplicación} \\
\hline
\textbf{Cerradura} & $a + b \in \mathbb{R}$ & $a \cdot b \in \mathbb{R}$ \\
\hline
\textbf{Conmutativa} & $a + b = b + a$ & $a \cdot b = b \cdot a$ \\
\hline
\textbf{Asociativa} & $a + (b + c) = (a + b) + c$ & $a \cdot (b \cdot c) = (a \cdot b) \cdot c$ \\
\hline
\textbf{Elemento neutro} & $a + 0 = a$ & $a \cdot 1 = a$ \\
\hline
\textbf{Inverso} & $a + (-a) = 0$ & $a \cdot \frac{1}{a} = 1$ (si $a \neq 0$) \\
\hline
\textbf{Distributiva} & $a(b + c) = ab + ac$ & N/A \\
\hline
\end{tabular}
\end{center}

\end{document}